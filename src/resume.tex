% resume.tex
% vim:set ft=tex spell:

\documentclass[10pt,letterpaper]{article}
\usepackage[letterpaper,margin=0.9in]{geometry}
\usepackage[utf8]{inputenc}
\usepackage{mdwlist}
\usepackage{hyperref}
\usepackage{fix-cm}
\usepackage[T1]{fontenc}
\usepackage{textcomp}
\usepackage{tgpagella}
\pagestyle{empty}
\setlength{\tabcolsep}{0em}

% indentsection style, used for sections that aren't already in lists
% that need indentation to the level of all text in the document
\newenvironment{indentsection}[1]%
{\begin{list}{}%
	{\setlength{\leftmargin}{#1}}%
	\item[]%
}
{\end{list}}

% opposite of above; bump a section back toward the left margin
\newenvironment{unindentsection}[1]%
{\begin{list}{}%
	{\setlength{\leftmargin}{-0.5#1}}%
	\item[]%
}
{\end{list}}

% format two pieces of text, one left aligned and one right aligned
\newcommand{\headerrow}[2]
{\begin{tabular*}{\linewidth}{l@{\extracolsep{\fill}}r}
	#1 &
	#2 \\
\end{tabular*}}

% make "C++" look pretty when used in text by touching up the plus signs
\newcommand{\CPP}
{C\nolinebreak[4]\hspace{-.05em}\raisebox{.22ex}{\footnotesize\bf ++}}

% and the actual content starts here
\begin{document}

\begin{center}
{\LARGE \textbf{Eric P. Stavarache}}

\ \ Zurich, Switzerland
\\
+40 770 770 753\ \ \textbullet
\ \ ericptst@gmail.com
\end{center}

\hrule
\vspace{-1em}

\subsection*{Education}

\begin{itemize}
	\parskip=0.1em

	\item
	\headerrow
		{\textbf{ETH Zurich}}
		{\textbf{Zurich, Switzerland}}
	\\
	\headerrow
		{\emph{Faculty of Computer Science, M.S. Computer Science}}
		{\emph{Expected 2020}}

	\item
	\headerrow
		{\textbf{University of Bucharest}}
		{\textbf{Bucharest, Romania}}
	\\
	\headerrow
		{\emph{Faculty of Mathematics and Informatics, B.S. Computer Science}}
		{\emph{June 2018}}

\end{itemize}

\hrule
\vspace{-1em}
\subsection*{Experience - Research}

\begin{itemize}
	\parskip=0.1em

	\item
	\headerrow
		{\textbf{ETH}}
		{\textbf{Zurich, CH}}
	\\
	\headerrow
		{\emph{Semester Project}}
		{\emph{February 2019 - June 2019}}
	\begin{itemize*}
        \item Research in Machine Learning, in the area of generative models,
            with the group of Angelika Steger.
        \item Looked at the disentanglement of Variational Auto Encoders (VAE)
            latent representation via multi-model systems.  The goal was for
            each model to learn to represent a single type of object, thanks to
            the KL-divergence loss term and to an incremental training process of the
            system.

            The system's goal was to recreate the input picture.
            Each model of the system would output a picture (what it thinks the
            reconstruction is), as well as a confidence mask (how sure it is
            that a given pixel it outputted is correct).
            The combined output of the system would then be a weighted
            combination of the individual models' output.

            The incremental training consisted of traning the first model only
            on the digit 1, and then the first two models on digits 1 and 2, and
            so on.
            The KL-divergence term in the VAE loss penalizes ``remembering""
            information, so the goal was that the optimal state would be for
            each model to remember a single digit, and to split the
            information evenly.
	\end{itemize*}

	\item
	\headerrow
		{\textbf{ETH}}
		{\textbf{Zurich, CH}}
	\\
	\headerrow
		{\emph{Semester Research Project}}
		{\emph{February 2019 - June 2019}}
	\begin{itemize*}
        \item Research in Machine Learning, in the area of natural language
            understanding, for a project under Elliott Ash.
        \item Looked into the connection between the semantic content of a legal
            document (i.e., the text it contains), and the network connections
            of a legal document (what other documents it references, and is
            referenced by).

            The ultimate goal was to predict possible links in the reference
            graph for a new, in-progress document, for which we would know the
            content, but which wouldn't have any references yet (because the
            lawyer writing it did not yet know them).

            We wanted to achieve this by embedding the training documents
            semantic-wise with word2vec, and graph-wise using node2vec, and then
            to find links either by projecting from the word2vec space into the
            node2vec space and then looking at nearest neighbours, or by
            training a link-prediction classifier.
	\end{itemize*}

	\item
	\headerrow
		{\textbf{ETH}}
		{\textbf{Zurich, CH}}
	\\
	\headerrow
		{\emph{Course Research Project}}
		{\emph{February 2019 - June 2019}}
	\begin{itemize*}
        \item Research in Theoretical Computer Science, in the area of
            distributed algorithms, for a project under Mohsen Ghaffari.
        \item Looked into a massively parallel algorithm (i.e., could scale to
            an arbitrary amount of machines) for computing the edit distance (ED)
            between two strings.

            The goal was to create a randomized sub-quadratic algorithm for
            approximating the ED.

            This was done by viewing the ED problem as a shortest path problem
            on the matrix of the classical dynamic programming approach.
            We then sample enough edges of this matrix-graph, such that
            computing the shortest path on the sampled edges gives us a constant
            approximation to the true distance.
	\end{itemize*}

	\item
	\headerrow
		{\textbf{ETH}}
		{\textbf{Zurich, CH}}
	\\
	\headerrow
		{\emph{Course Project}}
		{\emph{February 2019 - June 2019}}
	\begin{itemize*}
        \item Research in Software Engineering, in the area of
            high-performance computing, for a project under Markus Puschel.

        \item Built a high performance program for computing the Baum-Welch
            algorithm.

            Achieved performance near the theoretical maximum by optimizing the
            program for modern processor architectures.
            This involves re-organizing how the computation is done,
            to permit the usage of SIMD instructions, to achieve
            instruction-level parallelism, and to optimize memory accesses in
            accordance with the memory hierarchy of the computer.
	\end{itemize*}

\end{itemize}

\hrule
\vspace{-1em}
\subsection*{Experience - Work}

\begin{itemize}
	\parskip=0.1em

	\item
	\headerrow
		{\textbf{Google}}
		{\textbf{Zurich, CH}}
	\\
	\headerrow
		{\emph{Software Engineer Intern}}
		{\emph{October 2019 - April 2020}}
	\begin{itemize*}
        \item Worked in the assistant team to create a pipeline for generating
            plausible nicknames for common apps.

        \item Trained diverse ML models (linear regression, decision trees) to
            score and rank nicknames, using manually created datasets.

            Used statistical learning techniques (such as feature
            randomizaton and cross validation) to qualitatively asssess
            models.
	\end{itemize*}

	\item
	\headerrow
		{\textbf{Jane Street Capital}}
		{\textbf{London, UK}}
	\\
	\headerrow
		{\emph{Software Engineer Intern}}
		{\emph{July 2019 - September 2019}}
	\begin{itemize*}
		\item Distributed the computations of certain metrics on the cluster
            system, in OCaml.

        \item Created an open-source tool to analyze the code outputted by the
            OCaml compiler, in order to see which patterns of bytecode are most
            often generated. Available on
            \href{https://github.com/ericpts/ocaml-instr-freq}{github}.

        \item Found and fixed a register allocation bug in the OCaml compiler,
            which was causing it to generate inefficient code for deeply nested
            conditionals.

        \item Built a toy trading bot for the ETC challenge.
	\end{itemize*}

	\item
	\headerrow
		{\textbf{Google}}
		{\textbf{Mountain View, CA, US}}
	\\
	\headerrow
		{\emph{Software Engineer Intern}}
		{\emph{June 2018 - September 2018}}
	\begin{itemize*}
		\item Worked on a Java Bytecode Optimizer for Android, written in C++.
    \item Impacted Google Android applications by reducing their size.
	\end{itemize*}

	\item
	\headerrow
		{\textbf{Bloomberg}}
		{\textbf{London, UK}}
	\\
	\headerrow
		{\emph{Software Engineer Intern}}
		{\emph{July 2017 - September 2017}}
	\begin{itemize*}
		\item Built a Domain Specific Language in Python for integration testing.
		\item Worked on high-performance data storage in C++ with Redis for distributed storage.
	\end{itemize*}

	\item
	\headerrow
		{\textbf{Google}}
		{\textbf{New York, NY, US}}
	\\
	\headerrow
		{\emph{Software Engineer Intern}}
		{\emph{February 2017 - May 2017}}
	\begin{itemize*}
		\item Worked on a logging system for search, for monitoring systems' health.
		\item Implemented a full stack application with C++, Go and TypeScript with Preact.
	\end{itemize*}
\end{itemize}

\hrule
\vspace{-1em}

\subsection*{Experience - Volunteering \& \href{https://github.com/ericpts}{Projects}}

\vspace{-0.4em}

\begin{itemize*}

	\item Held an advanced algorithms class for outstanding High School students.

	\item
		{Member of the scientific committees of various algorithmic contests}

	\headerrow
		{\textbf{Central European Olympiad, Romanian Olympiad}}
		{Ministry of Education}

\vspace{-0.6em}

	\begin{itemize*}
		\item Proposed tasks, helped with writing statements \& implementing solutions.
	\end{itemize*}

\vspace{-0.6em}

    \item Built an interpretor for a Scheme-like language in Java.
    \item Implemented multiple Computer Vision projects such as object detection with ML, texture extraction or content aware image resizing.
\end{itemize*}



\hrule

\vspace{-1em}

\subsection*{Experience - Competitive Programming}
\begin{itemize}
	\parskip=0.1em
		\item
		\headerrow
			{14th place out of 88 international teams at the ACM SEERC}
			{2017}

		\item
		\headerrow
			{12th place out of 50 at the Google Hash Code Finals in Paris}
            {\href{https://hashcode.withgoogle.com/hashcode_2016.html}{2016}}

        \item
		\headerrow
            {\textbf{Gold} and \textbf{Silver} medals at the Romanian Olympiad}
			{2013, 2015}

        \item
		\headerrow
            {Participant in International Olympiad selection camps}
			{2013, 2015}

\end{itemize}

\hrule
\vspace{-1em}

\subsection*{Technical Skills}

\begin{indentsection}{\parindent}
\hyphenpenalty=1000
\begin{description*}
	\item
	{\textbf{Languages}}

	\begin{itemize*}
        \item {Advanced: C++, Python, Bash}
        \item {Intermediate: Go, JavaScript/TypeScript, Rust, Java, Scheme}
	\end{itemize*}

	\item
	{\textbf{Theoretical Computer Science}}

	Algorthims, Data Structures, Cryptography, Computer Vision, Artificial Intelligence, Automata theory
\end{description*}
\end{indentsection}

\end{document}

